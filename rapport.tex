\documentclass[a4paper]{article}
\usepackage{float}
\usepackage[utf8]{inputenc}
\usepackage[T1]{fontenc}
\usepackage{graphicx}
\usepackage[frenchb]{babel}
\usepackage{amsmath}
\usepackage{listings}
\usepackage{hyperref}

% define our color
\usepackage{xcolor}

% code color
\definecolor{ligthyellow}{RGB}{250,247,220}
\definecolor{darkblue}{RGB}{5,10,85}
\definecolor{ligthblue}{RGB}{1,147,128}
\definecolor{darkgreen}{RGB}{8,120,51}
\definecolor{darkred}{RGB}{160,0,0}

% other color
\definecolor{ivi}{RGB}{141,107,185}


\lstset{
    language=scilab,
    captionpos=b,
    extendedchars=true,
    frame=lines,
    numbers=left,
    numberstyle=\tiny,
    numbersep=5pt,
    keepspaces=true,
    breaklines=true,
    showspaces=false,
    showstringspaces=false,
    breakatwhitespace=false,
    stepnumber=1,
    showtabs=false,
    tabsize=3,
    basicstyle=\small\ttfamily,
    backgroundcolor=\color{ligthyellow},
    keywordstyle=\color{ligthblue},
    morekeywords={include, printf, uchar},
    identifierstyle=\color{darkblue},
    commentstyle=\color{darkgreen},
    stringstyle=\color{darkred},
}

\begin{document}

\title{M3DA -- Rapport Intégrations Implicites \& Explicites}
\author{Camus Tristan, Cojez Arnaud}
\date{jeudi 5 janvier 2017}

\maketitle

%----------------------------------------------------------------------------------------
%	INTRODUCTION
%----------------------------------------------------------------------------------------

\section{Introduction}

Dans ce rapport nous nous intéresserons aux méthodes d'intégrations de forces. Nous aborderons les modèles masse-ressort en intégration explicite et implicite, ainsi que la gestion de contacts.

\section{Intégration explicite}

Pour implémenter un modèle massee-ressort avec intégration temporelle explicite, il faut définir l'accélération et la vitesse des points et utiliser le schéma d'intégration {\em Euler explicite}.

\subsection{Euler explicite}

La position d'un point à l'instant t correspond à :
\begin{equation}
  p_t = p_{t-1} + v_t * dt
\end{equation}
où $p_t =$ position du point à l'instant t, $p_{t-1} =$ position du point à l'instant t-1, $v_t =$ vitesse du point à l'instant t, $dt =$ pas de temps.\\

La vitesse d'un point à l'instant t correspond à :
\begin{equation}
  v_t = v_{t-1} + a_t * dt
\end{equation}
où $v_t =$ vitesse du point à l'instant t, $v_{t-1} =$ vitesse du point à l'instant t-1, $a_t =$ accéleration du point à l'instant t, $dt =$ pas de temps.\\

L'accelération d'un point à l'instant t correspond à :
\begin{equation}
  a_t = \frac{\sum F}{m}
\end{equation}
où $\sum F$ correspond à la somme des forces appliquées sur le point et $m$ correspond à la masse de l'objet représenté par le point.

\subsection{Cas de la chute libre}

Dans le cas de la chute libre, comme il n'y a que la gravité qui s'applique sur l'objet, on peut simplifier l'équation $a_t = \sum \frac{F}{m}$ en :
\begin{equation}
  a_t = \frac{P}{m} = \frac{G * m}{m} = G
\end{equation}
où $P$ correspond au poids de l'objet, dont la formule est $G * m$, où $G$ correspond à la valeur de la gravité (en général 9,81) et $m$ correspond à la masse de l'objet.

\subsection{Ajout de ressorts}

L'ajout de ressorts permet à l'objet de se déformer lorsqu'il tombe en ayant des points fixes.\\

La somme des forces appliquées à un point correspond ici à :
\begin{equation}
  a_t = \frac{P + R}{m}
\end{equation}
où $P$ correspond au poids de l'objet, $R$ correspond à la force du ressort, $m$ la masse de l'objet.

Pour obtenir la force appliquée par le ressort sur le point, il nous faut la valeur nominale de la force, et la direction dans laquelle la force est orientée.\\

La valeur nominale de la force correspond à :
\begin{equation}
  vn_R = k * (l - l_0)
\end{equation}
où $k$ correspond au coefficient de résistance du ressort, $l$ correspond à la longueur actuelle du ressort et $l_0$ sa longueur initiale.\\

La direction (ou orientation) de la force correspond à :
\begin{equation}
  d = \frac{pos_B - pos_A}{norm(pos_B - pos_A)}
\end{equation}
où $pos_X$ correspond à la position du point X, et $norm(vect)$ correspond au vecteur vect normalisé.\\

La force appliquée par le ressort sur le point correspond à :
\begin{equation}
  R = vn_R * d
\end{equation}
où $vn_R$ correspond à la force nominale (voir ci-dessus) et d correspond à l'orientation de la force (voir ci-dessus).\\

L'accélération du point à l'instant t correspond donc à :
\begin{equation}
  a_t = \frac{P + R}{m} = \frac{P}{m} + \frac{R}{m} = g + \frac{vn_R * d}{m}
\end{equation}

Il faut faire attention à calculer et conserver la valeur initiale de la longueur d'un segment/ressort, et recalculer à chaque pas de temps la longueur actuelle du ressort.

\subsection{Observations}

On observe que si l'on fixe un des bords du solide, une déformation se propage le long de celui-ci pendant la chute.\\

En faisant varier le pas de temps $dt$, on constate qu'il ne doit pas être trop grand, sinon l'objet "explose".

%----------------------------------------------------------------------------------------
\section{Intégration implicite}

L'intégration implicite fonctionne d'une façon similaire à l'intégration explicite. Il faut appliquer ici le schéma {\em Euler implicite}. L'intérêt est d'avoir une simulation plus stable (pas "d'explosion").\\

\subsection{Euler implicite}

La position et la vitesse sont calculées de la même façon que dans {\em Euler explicite}.\\

L'accelération d'un point à l'instant t correspond à :
\begin{equation}
  m * a_t = m * G + Fk(p_t)
\end{equation}
où $G$ correspond à la valeur de la gravité (en général 9,81) et $m$ correspond à la masse de l'objet. $Fk$ correspond à une fonction qui calcule la force générée par les ressorts à partir de la position.\\

L'accélération du point à l'instant t correspond également à :
\begin{equation}
  a_t = \frac{dv}{dt} = v_t - v_{t-1}
\end{equation}
où $dv =$ variable commune liant l'accélération et la position, $dt =$ pas de temps, $v_t =$ vitesse du point à l'instant t, $v_{t-1} =$ vitesse du point à l'instant t-1.\\

La position du point à l'instant t correspond également à :
\begin{equation}
  p_t = p_{t-1} + v_{t-1} + dv
\end{equation}
où $p_{t-1} =$ position du point à l'instant t-1, $v_{t-1} =$ vitesse du point à l'instant t-1, $dv =$ variable commune liant l'accélération et la position.\\

On dispose donc d'une fonction non-linéaire :
\begin{equation}
  Fnl(dv) = \frac{m * dv}{dt} - m * G - Fk( p_{t-1} + v_{t-1} + dv )
\end{equation}
où $m$ correspond à la masse de l'objet, $dv =$ variable commune liant l'accélération et la position, $dt =$ pas de temps, $G$ correspond à la valeur de la gravité (en général 9,81), $Fk$ correspond à une fonction qui calcule la force générée par les ressorts à partir de la position, $p_{t-1} =$ position du point à l'instant t-1, $v_{t-1} =$ vitesse du point à l'instant t-1.\\

Pour résoudre cette équation, il faut utiliser un solveur.
\begin{equation}
  dv = fsolve(dv_0, Fnl)
\end{equation}
où fsolve implémente le solveur (dans Scilab), $dv_0$ représente une première estimation de dv, et $Fnl$ correspond à la fonction non-linéaire trouvée plus haut.

\subsection{Observations}

On utilise un solveur pour résoudre l'équation non-linéaire $Fnl$.
Par conséquent, l'équation prend plus de temps à résoudre.\\
Il faut donc sacrifier les performances afin d'avoir un système plus stable.\\

Le choix du pas de temps influence donc peu sur le résultat. De même une modification de la raideur n'a pas d'impact sur la stabilité du système.

%----------------------------------------------------------------------------------------
\section{Gestion des contacts}

On veut implémenter une réponse à la collision entre un objet en chute libre et un plan représentant une surface rigide et statique.\\

Une implémentation possible est celle de la méthode de pénalité. Cette méthode est utilisable avec les intégrations implicites et explicites.\\

Une pénalité correspond à une force appliquée sur un objet pour le repousser, lorsqu'une collision est détectée (dépassement du plan dans notre cas).

\subsection{Pénalités}

On calcule la masse du disque calculée, puis on applique la force de gravité sur celui-ci.\\

Une fois que l'on détecte que l'objet dépasse le plan, on ajoute la force de pénalité :
\begin{equation}
  Pe = -k * distance(p_X, plan_X)
\end{equation}
où $k$ correspond au coefficient de résistance du ressort, $p_X$ correspond à la position du point X, et $plan_X$ correspond à la projection orthogonale du point X sur le plan.\\

Pour éviter à l'objet de rebondir de façon infinie, on ajoute à cette force un amortissement :
\begin{equation}
  Pe = -k * distance(p_X, plan_X) - b * v_{t-1}
\end{equation}
où $b$ correspond au coefficient de frottement, et $v_{t-1}$ correspond à la vitesse du point à l'instant t-1.\\

Cette force sera ajoutée aux forces utilisées dans les intégrations vues précédemments.

\subsection{Observations}

On constate que le disque rebondit sur le plan comme prévu.\\
Lorsque l'on augmente le coefficient d'amortissement, le disque rebondit moins haut, et inversement.

\end{document}
