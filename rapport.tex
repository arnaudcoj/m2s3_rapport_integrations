\documentclass[a4paper]{article}
\usepackage{float}
\usepackage[utf8]{inputenc}
\usepackage[T1]{fontenc}
\usepackage{graphicx}
\usepackage[frenchb]{babel}
\usepackage{amsmath}
\usepackage{listings}
\usepackage{hyperref}

% define our color
\usepackage{xcolor}

% code color
\definecolor{ligthyellow}{RGB}{250,247,220}
\definecolor{darkblue}{RGB}{5,10,85}
\definecolor{ligthblue}{RGB}{1,147,128}
\definecolor{darkgreen}{RGB}{8,120,51}
\definecolor{darkred}{RGB}{160,0,0}

% other color
\definecolor{ivi}{RGB}{141,107,185}


\lstset{
    language=scilab,
    captionpos=b,
    extendedchars=true,
    frame=lines,
    numbers=left,
    numberstyle=\tiny,
    numbersep=5pt,
    keepspaces=true,
    breaklines=true,
    showspaces=false,
    showstringspaces=false,
    breakatwhitespace=false,
    stepnumber=1,
    showtabs=false,
    tabsize=3,
    basicstyle=\small\ttfamily,
    backgroundcolor=\color{ligthyellow},
    keywordstyle=\color{ligthblue},
    morekeywords={include, printf, uchar},
    identifierstyle=\color{darkblue},
    commentstyle=\color{darkgreen},
    stringstyle=\color{darkred},
}

\begin{document}

\title{M3DA -- Rapport Intégrations Implicites \& Explicites}
\author{Camus Tristan, Cojez Arnaud}
\date{jeudi 5 janvier 2017}

\maketitle

%----------------------------------------------------------------------------------------
%	INTRODUCTION
%----------------------------------------------------------------------------------------

\section{Introduction}

Dans ce rapport nous nous intéresserons aux méthodes d'intégrations de forces. Nous aborderons les modèles masse-ressort en intégration explicite et implicite, ainsi que la gestion de contacts.

\clearpage
\section{bordel}

on a compris que 

dt est le pas de temps
la position correspond à la position initiale + la vitesse * dt
la vitesse est augmentée à chaque tour par l'acceleration * dt
L'acceleration correspond à la somme des forces divisée par la masse

La somme des forces appliquées à un point correspond ici au poids + la force du ressort
Poids = m * g

valeurNominaleForce = k * (longueur - longueurInitiale)
direction = (pos_b - pos_a) / norm(pos_b - pos_a)
k = coefficient de résistance du ressort
Force_ressort = valeurNominaleForce * direction
on doit donc calculer et conserver la valeur initiale de la longueur d'un segment/ressort, puis à chaque pas de temps recalculer la longueur

on constate une propagation numérique de la déformation due au ressort
Il ne faut pas mettre un dt trop grand sinon la force explose

fonction non linéaire prend dv en paramètre. Utilise un solveur pour résoudre l'équation.
Prend plus de temps à résoudre mais plus stable

choix du pas de temps influence peu sur le résultat
peut augmenter constantes comme la raideur et reste stable

pénalité : quand dépasse une limite (devrait y'avoir une collision), on applique une force qui repousse l'objet

on peut le faire en explicite ou implicite

pour l'explicite on fait un cas où c'est rigide
on a calculé la masse du disque puis appliqué une force de gravité sur tous les points du maillage
puis une fois qu'on détecte que l'objet a dépassé le plan on ajoute la force de pénalité FC += -k * distance_du_point_au_plan
On intègre ça de façon explicite comme vu au premier tp

On ajoute à ça un amortissement car rebondit à l'infini sinon : FC += -k * distance_point_au_plan - v * vitesse

%----------------------------------------------------------------------------------------

\end{document}